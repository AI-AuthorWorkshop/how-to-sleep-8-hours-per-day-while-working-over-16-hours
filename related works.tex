\section{Related Works}

\paragraph{Sleep Hygiene and Sleep Quality}
Several studies have investigated the factors affecting sleep quality and the role of sleep hygiene practices in improving sleep. \citet{lin2022chronotype} found that chronotype and trait self-control positively predicted sleep quality, with sleep hygiene habits and bedtime media use mediating the relationship. However, the study focused on Chinese adults and may not be generalizable to other populations. \citet{manzar2021anxiety} reported a high prevalence of anxiety symptoms in university students, which were associated with poor sleep and inadequate sleep hygiene parameters. This study highlights the importance of addressing mental health and its diverse sleep correlates in university students. \citet{humphries2021dysfunctional} suggested that sleep interventions for university students should focus on common hygiene components, such as using the bed for activities other than sleeping and managing thinking and worrying before bed. \citet{molla2021magnitude} found that almost half of the medical students had poor sleep hygiene practice, emphasizing the need for routine screening of depressive and stress symptoms and education about sleep hygiene among medical students. \citet{zhang2020predicting} indicated that college students' hand washing and sleep hygiene behaviors are a function of both motivational and volitional factors, providing insights into the factors influencing sleep hygiene practices.

\paragraph{Time Management and Academic Performance}
Effective time management has been shown to be associated with better academic performance and well-being. \citet{aeon2021does} found that time management is moderately related to job performance, academic achievement, and well-being, with women's time management scores increasing over the past few decades. \citet{adams2019impact} reported that students' perceived control of time correlated significantly with cumulative grade point average, but no significant differences were found across gender, age, entry qualification, and time spent in the program. \citet{uzir2019analytics} demonstrated that meaningful and theoretically relevant time management patterns can be detected from trace data as manifestations of students' tactics and strategies, with time management tactics having significant associations with academic performance.

\paragraph{Work-Life Balance and Employee Performance}
The relationship between work-life balance and employee performance has been explored in various studies. \citet{mulang2022analysis} found that employee engagement could not mediate the effect of organizational justice and work-life balance on turnover intention, but the direct effect showed positive and significant results from the two independent variables on the dependent variable. \citet{arifin2022pengaruh} reported that work-life balance had a positive and significant effect on employee performance, with the total effect of the variable amounting to 85.1\%. \citet{alsyah2022pengaruh} found that non-physical work environment and work-life balance had a partial and simultaneous effect on employee performance at the Office of Youth, Sports, and Tourism in Pati Regency, with a total effect of 97\%. \citet{fardiani2022the} examined the role of leader-member exchange (LMX) in mediating the contribution of work-life balance and work engagement on employee organizational commitment, finding that LMX partially mediated the contribution of work-life balance and fully mediated the contribution of work engagement to changes in commitment.